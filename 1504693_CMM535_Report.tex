\documentclass[10pt  ,usenames, dvipsnames]{article}\usepackage[]{graphicx}\usepackage[]{color}
%% maxwidth is the original width if it is less than linewidth
%% otherwise use linewidth (to make sure the graphics do not exceed the margin)
\makeatletter
\def\maxwidth{ %
  \ifdim\Gin@nat@width>\linewidth
    \linewidth
  \else
    \Gin@nat@width
  \fi
}
\makeatother

\definecolor{fgcolor}{rgb}{0.345, 0.345, 0.345}
\newcommand{\hlnum}[1]{\textcolor[rgb]{0.686,0.059,0.569}{#1}}%
\newcommand{\hlstr}[1]{\textcolor[rgb]{0.192,0.494,0.8}{#1}}%
\newcommand{\hlcom}[1]{\textcolor[rgb]{0.678,0.584,0.686}{\textit{#1}}}%
\newcommand{\hlopt}[1]{\textcolor[rgb]{0,0,0}{#1}}%
\newcommand{\hlstd}[1]{\textcolor[rgb]{0.345,0.345,0.345}{#1}}%
\newcommand{\hlkwa}[1]{\textcolor[rgb]{0.161,0.373,0.58}{\textbf{#1}}}%
\newcommand{\hlkwb}[1]{\textcolor[rgb]{0.69,0.353,0.396}{#1}}%
\newcommand{\hlkwc}[1]{\textcolor[rgb]{0.333,0.667,0.333}{#1}}%
\newcommand{\hlkwd}[1]{\textcolor[rgb]{0.737,0.353,0.396}{\textbf{#1}}}%
\let\hlipl\hlkwb

\usepackage{framed}
\makeatletter
\newenvironment{kframe}{%
 \def\at@end@of@kframe{}%
 \ifinner\ifhmode%
  \def\at@end@of@kframe{\end{minipage}}%
  \begin{minipage}{\columnwidth}%
 \fi\fi%
 \def\FrameCommand##1{\hskip\@totalleftmargin \hskip-\fboxsep
 \colorbox{shadecolor}{##1}\hskip-\fboxsep
     % There is no \\@totalrightmargin, so:
     \hskip-\linewidth \hskip-\@totalleftmargin \hskip\columnwidth}%
 \MakeFramed {\advance\hsize-\width
   \@totalleftmargin\z@ \linewidth\hsize
   \@setminipage}}%
 {\par\unskip\endMakeFramed%
 \at@end@of@kframe}
\makeatother

\definecolor{shadecolor}{rgb}{.97, .97, .97}
\definecolor{messagecolor}{rgb}{0, 0, 0}
\definecolor{warningcolor}{rgb}{1, 0, 1}
\definecolor{errorcolor}{rgb}{1, 0, 0}
\newenvironment{knitrout}{}{} % an empty environment to be redefined in TeX

\usepackage{alltt}
\usepackage{graphicx, verbatim}
\usepackage{amsmath}
\usepackage{amssymb}
\usepackage{amscd}
\usepackage{lipsum}
\usepackage{todonotes}
\usepackage[tableposition=top]{caption}
\usepackage{ifthen}
\usepackage[utf8]{inputenc}
\usepackage{graphicx}
\usepackage{caption}
\usepackage{listings}
\usepackage{color}
\setlength{\textwidth}{6.5in} 
\setlength{\textheight}{9in}
\setlength{\oddsidemargin}{0in} 
\setlength{\evensidemargin}{0in}
\setlength{\topmargin}{-1.5cm}
\setlength{\parindent}{0cm}
\usepackage{setspace}
\usepackage{float}
\usepackage{amssymb}
\usepackage[utf8]{inputenc}
\usepackage{fancyhdr}
\usepackage{tabularx}
\usepackage{lmodern} % for bold teletype font
\usepackage{minted}

\usepackage{hyperref}
\hypersetup{
  colorlinks   = true, %Colours links instead of ugly boxes
  urlcolor     = blue, %Colour for external hyperlinks
  linkcolor    = blue, %Colour of internal links
  citecolor   = red %Colour of citations
}

%\fancyhf{}
\rfoot{Andrew Tait \thepage}
\singlespacing
\usepackage[affil-it]{authblk} 
\usepackage{etoolbox}
\usepackage{lmodern}

% Notice the following package, it will help you cite papers
\usepackage[backend=bibtex ,sorting=none]{biblatex}
\bibliography{references}

\begin{filecontents*}{references.bib}

\end{filecontents*}
\IfFileExists{upquote.sty}{\usepackage{upquote}}{}
\begin{document}


\title{\LARGE Coursework  \\ Data Science Development (CMM535)}

\author{Andrew Tait, \textit{\href{1504693@rgu.ac.uk}{1504693@rgu.ac.uk}}}
\maketitle
% \begin{flushleft} \today \end{flushleft} 
\noindent\rule{16cm}{0.4pt}
%\underline{\hspace{3cm}
\ \\
%\thispagestyle{empty}



\section {Data Exploration}



\subsection{Dataset Choice}
The dataset that has been chosen for this part of the coursework is Mushroom. This is available on the UCI repository. The set was chosen because of it's adequate instance size and number of attributes.

.


\subsection{Problem Statement and Data Exploration}


The main purpose of the Mushroom dataset is to identify which characteristics (attributes) determine if a particular mushroom species is editable or poisonous.

Therefore the aim of this assignment is to build a predictive model to predict if a certain type of Mushroom is ediable or not.


To start off the data explortation I will first import the required librarys.

\begin{knitrout}
\definecolor{shadecolor}{rgb}{0.969, 0.969, 0.969}\color{fgcolor}\begin{kframe}
\begin{alltt}
\hlcom{#Import packages}
\hlkwd{library}\hlstd{(randomForest)}
\hlkwd{library}\hlstd{(e1071)}
\hlkwd{library}\hlstd{(caret)}
\hlkwd{library}\hlstd{(ggplot2)}
\hlkwd{library}\hlstd{(gridExtra)}
\hlkwd{library}\hlstd{(caret)}
\hlkwd{library}\hlstd{(rpart.plot)}
\hlkwd{library}\hlstd{(RColorBrewer)}
\hlkwd{library}\hlstd{(plyr)}
\end{alltt}
\end{kframe}
\end{knitrout}

\begin{knitrout}
\definecolor{shadecolor}{rgb}{0.969, 0.969, 0.969}\color{fgcolor}\begin{kframe}
\begin{alltt}
\hlcom{#Import packages}
\hlkwd{library}\hlstd{(randomForest)}
\end{alltt}


{\ttfamily\noindent\itshape\color{messagecolor}{randomForest 4.6-12}}

{\ttfamily\noindent\itshape\color{messagecolor}{Type rfNews() to see new features/changes/bug fixes.}}\begin{alltt}
\hlkwd{library}\hlstd{(e1071)}
\hlkwd{library}\hlstd{(caret)}
\end{alltt}


{\ttfamily\noindent\itshape\color{messagecolor}{Loading required package: lattice}}

{\ttfamily\noindent\itshape\color{messagecolor}{Loading required package: ggplot2}}

{\ttfamily\noindent\itshape\color{messagecolor}{\\Attaching package: 'ggplot2'}}

{\ttfamily\noindent\itshape\color{messagecolor}{The following object is masked from 'package:randomForest':

\ \ \ \ margin}}\begin{alltt}
\hlkwd{library}\hlstd{(ggplot2)}
\hlkwd{library}\hlstd{(gridExtra)}
\end{alltt}


{\ttfamily\noindent\itshape\color{messagecolor}{\\Attaching package: 'gridExtra'}}

{\ttfamily\noindent\itshape\color{messagecolor}{The following object is masked from 'package:randomForest':

\ \ \ \ combine}}\begin{alltt}
\hlkwd{library}\hlstd{(caret)}
\hlkwd{library}\hlstd{(rpart.plot)}
\end{alltt}


{\ttfamily\noindent\bfseries\color{errorcolor}{Error in library(rpart.plot): there is no package called 'rpart.plot'}}\begin{alltt}
\hlkwd{library}\hlstd{(RColorBrewer)}
\hlkwd{library}\hlstd{(plyr)}
\end{alltt}
\end{kframe}
\end{knitrout}


Then set the working directory to the Coursework project folder path:
\begin{knitrout}
\definecolor{shadecolor}{rgb}{0.969, 0.969, 0.969}\color{fgcolor}\begin{kframe}
\begin{alltt}
\hlkwd{setwd}\hlstd{(}\hlstr{"~/CMM535 Data Science Development/Coursework/CMM535_Coursework"}\hlstd{)}
\end{alltt}
\end{kframe}
\end{knitrout}

In order to import the dataset, I used a third party helper function, which can be viewed at Appdenix 1.
The helper function not only set the attributes names but the instances names as well. Since all the data is represented a single character, it converts them into their string equivalent.


\begin{knitrout}
\definecolor{shadecolor}{rgb}{0.969, 0.969, 0.969}\color{fgcolor}\begin{kframe}
\begin{alltt}
\hlcom{#helper function}
\hlkwd{source}\hlstd{(}\hlstr{'helper_functions.r'}\hlstd{)}

\hlcom{#Import datasets using helper function}

\hlstd{mushroom} \hlkwb{<-} \hlkwd{fetchAndCleanData}\hlstd{()}
\end{alltt}
\end{kframe}
\end{knitrout}


\begin{knitrout}
\definecolor{shadecolor}{rgb}{0.969, 0.969, 0.969}\color{fgcolor}\begin{kframe}
\begin{alltt}
\hlcom{#helper function}
\hlkwd{source}\hlstd{(}\hlstr{'helper_functions.r'}\hlstd{)}

\hlcom{#Import datasets using helper function}

\hlstd{mushroom} \hlkwb{<-} \hlkwd{fetchAndCleanData}\hlstd{()}
\end{alltt}
\end{kframe}
\end{knitrout}


Now that the dataset is imported, it is time to do some data exploration and analysis.

Number of rows in the dataset

\begin{knitrout}
\definecolor{shadecolor}{rgb}{0.969, 0.969, 0.969}\color{fgcolor}\begin{kframe}
\begin{alltt}
\hlcom{#Number of rows in the dataset}
\hlkwd{nrow}\hlstd{(mushroom)}
\end{alltt}
\end{kframe}
\end{knitrout}


\begin{knitrout}
\definecolor{shadecolor}{rgb}{0.969, 0.969, 0.969}\color{fgcolor}\begin{kframe}
\begin{verbatim}
## [1] 8124
\end{verbatim}
\end{kframe}
\end{knitrout}

Number of columns (features) in the dataset

\begin{knitrout}
\definecolor{shadecolor}{rgb}{0.969, 0.969, 0.969}\color{fgcolor}\begin{kframe}
\begin{alltt}
\hlkwd{ncol}\hlstd{(mushroom)}
\end{alltt}
\end{kframe}
\end{knitrout}


\begin{knitrout}
\definecolor{shadecolor}{rgb}{0.969, 0.969, 0.969}\color{fgcolor}\begin{kframe}
\begin{verbatim}
## [1] 23
\end{verbatim}
\end{kframe}
\end{knitrout}

Summary of the Mushroom dataset

\begin{knitrout}
\definecolor{shadecolor}{rgb}{0.969, 0.969, 0.969}\color{fgcolor}\begin{kframe}
\begin{alltt}
\hlcom{#Summar of the Mushroom dataset}
\hlkwd{str}\hlstd{(mushroom)}
\end{alltt}
\end{kframe}
\end{knitrout}


\begin{knitrout}
\definecolor{shadecolor}{rgb}{0.969, 0.969, 0.969}\color{fgcolor}\begin{kframe}
\begin{verbatim}
## 'data.frame':	8124 obs. of  23 variables:
##  $ Edible               : Factor w/ 2 levels "Edible","Poisonous": 2 1 1 2 1 1 1 1 2 1 ...
##  $ CapShape             : Factor w/ 12 levels "b","c","f","k",..: 9 9 7 9 9 9 7 7 9 7 ...
##  $ CapSurface           : Factor w/ 8 levels "f","g","s","y",..: 8 8 8 7 8 7 8 7 7 8 ...
##  $ CapColor             : Factor w/ 20 levels "b","c","e","g",..: 11 20 19 19 14 20 19 19 19 20 ...
##  $ Bruises              : Factor w/ 4 levels "f","t","True",..: 3 3 3 3 4 3 3 3 3 3 ...
##  $ Odor                 : Factor w/ 18 levels "a","c","f","l",..: 17 10 11 17 16 10 10 11 17 10 ...
##  $ GillAttachment       : Factor w/ 6 levels "a","f","Attached",..: 5 5 5 5 5 5 5 5 5 5 ...
##  $ GillSpacing          : Factor w/ 5 levels "c","w","Close",..: 3 3 3 3 4 3 3 3 3 3 ...
##  $ GillSize             : Factor w/ 4 levels "b","n","Broad",..: 4 3 3 4 3 3 3 3 4 3 ...
##  $ GillColor            : Factor w/ 24 levels "b","e","g","h",..: 13 13 14 14 13 14 17 14 20 17 ...
##  $ StalkShape           : Factor w/ 4 levels "e","t","Enlarging",..: 3 3 3 3 4 3 3 3 3 3 ...
##  $ StalkRoot            : Factor w/ 12 levels "?","b","c","e",..: 9 7 7 9 9 7 7 7 9 7 ...
##  $ StalkSurfaceAboveRing: Factor w/ 8 levels "f","k","s","y",..: 8 8 8 8 8 8 8 8 8 8 ...
##  $ StalkSurfaceBelowRing: Factor w/ 8 levels "f","k","s","y",..: 8 8 8 8 8 8 8 8 8 8 ...
##  $ StalkColorAboveRing  : Factor w/ 18 levels "b","c","e","g",..: 17 17 17 17 17 17 17 17 17 17 ...
##  $ StalkColorBelowRing  : Factor w/ 18 levels "b","c","e","g",..: 17 17 17 17 17 17 17 17 17 17 ...
##  $ VeilType             : Factor w/ 3 levels "p","Partial",..: 2 2 2 2 2 2 2 2 2 2 ...
##  $ VeilColor            : Factor w/ 8 levels "n","o","w","y",..: 7 7 7 7 7 7 7 7 7 7 ...
##  $ RingNumber           : Factor w/ 6 levels "n","o","t","None",..: 5 5 5 5 5 5 5 5 5 5 ...
##  $ RingType             : Factor w/ 13 levels "e","f","l","n",..: 11 11 11 11 7 11 11 11 11 11 ...
##  $ SporePrintColor      : Factor w/ 18 levels "b","h","k","n",..: 10 11 11 10 11 10 10 11 10 10 ...
##  $ Population           : Factor w/ 12 levels "a","c","n","s",..: 10 9 9 10 7 9 9 10 11 10 ...
##  $ Habitat              : Factor w/ 14 levels "d","g","l","m",..: 12 8 10 12 8 8 10 10 8 10 ...
\end{verbatim}
\end{kframe}
\end{knitrout}

Now that some basic data exploration is covered, next to inspect the dataset a bit further. Starting with the class (edidable) distribution in the mushroom dataset,

\begin{knitrout}
\definecolor{shadecolor}{rgb}{0.969, 0.969, 0.969}\color{fgcolor}\begin{kframe}
\begin{alltt}
\hlcom{#Class Distribution}
\hlkwd{barplot}\hlstd{(}\hlkwd{table}\hlstd{(mushroom}\hlopt{$}\hlstd{Edible))}
\end{alltt}
\end{kframe}
\end{knitrout}

\begin{knitrout}
\definecolor{shadecolor}{rgb}{0.969, 0.969, 0.969}\color{fgcolor}
\includegraphics[width=.76\linewidth]{figure/unnamed-chunk-13-1} 

\end{knitrout}


Next is to analyse if there is a correlocation between the CapShape and CapSurface of a mushroom and whether it isEdible or Poisonous. Which is shown in the plot below.

\begin{knitrout}
\definecolor{shadecolor}{rgb}{0.969, 0.969, 0.969}\color{fgcolor}\begin{kframe}
\begin{alltt}
\hlcom{#Comparisons of CapShape and CapSurface with Edible or Poisonous}
\hlkwd{ggplot}\hlstd{(mushroom,}\hlkwd{aes}\hlstd{(}\hlkwc{x}\hlstd{=CapShape,} \hlkwc{y}\hlstd{=CapSurface,} \hlkwc{color}\hlstd{=Edible))} \hlopt{+}
                       \hlkwd{geom_jitter}\hlstd{(}\hlkwc{alpha}\hlstd{=}\hlnum{0.3}\hlstd{)} \hlopt{+}
                       \hlkwd{scale_color_manual}\hlstd{(}\hlkwc{breaks} \hlstd{=} \hlkwd{c}\hlstd{(}\hlstr{'Edible'}\hlstd{,}\hlstr{'Poisonous'}\hlstd{),}
                                          \hlkwc{values}\hlstd{=}\hlkwd{c}\hlstd{(}\hlstr{'darkgreen'}\hlstd{,}\hlstr{'red'}\hlstd{))}
\end{alltt}
\end{kframe}
\end{knitrout}
\centering
\caption{Mushroom Data}
\begin{knitrout}
\definecolor{shadecolor}{rgb}{0.969, 0.969, 0.969}\color{fgcolor}
\includegraphics[width=.76\linewidth]{figure/unnamed-chunk-15-1} 

\end{knitrout}
\label{tab:example}



\clearpage


% latex table generated in R 3.4.2 by xtable 1.8-2 package
% Tue Apr 17 23:21:55 2018
\begin{table}[ht]
\centering
\caption{Mushroom Data}
\scalebox{0.75}{
\begin{tabular}{p{1in}|c|c|c|c|c|c|c|c|c|c|c|c|c|c|c|c|c|c|c|c|c|c|c}
  \hline
 & Edible & CapShape & CapSurface & CapColor & Bruises & Odor & GillAttachment & GillSpacing & GillSize & GillColor & StalkShape & StalkRoot & StalkSurfaceAboveRing & StalkSurfaceBelowRing & StalkColorAboveRing & StalkColorBelowRing & VeilType & VeilColor & RingNumber & RingType & SporePrintColor & Population & Habitat \\ 
  \hline
1 & Poisonous & Convex & Smooth & Brown & True & Pungent & Free & Close & Narrow & Black & Enlarging & Equal & Smooth & Smooth & White & White & Partial & White & One & Pendant & Black & Scattered & Urban \\ 
  2 & Edible & Convex & Smooth & Yellow & True & Almond & Free & Close & Broad & Black & Enlarging & Club & Smooth & Smooth & White & White & Partial & White & One & Pendant & Brown & Numerous & Grasses \\ 
  3 & Edible & Bell & Smooth & White & True & Anise & Free & Close & Broad & Brown & Enlarging & Club & Smooth & Smooth & White & White & Partial & White & One & Pendant & Brown & Numerous & Meadows \\ 
  4 & Poisonous & Convex & Scaly & White & True & Pungent & Free & Close & Narrow & Brown & Enlarging & Equal & Smooth & Smooth & White & White & Partial & White & One & Pendant & Black & Scattered & Urban \\ 
  5 & Edible & Convex & Smooth & Gray & False & None & Free & Crowded & Broad & Black & Tapering & Equal & Smooth & Smooth & White & White & Partial & White & One & Evanescent & Brown & Abundnant & Grasses \\ 
  6 & Edible & Convex & Scaly & Yellow & True & Almond & Free & Close & Broad & Brown & Enlarging & Club & Smooth & Smooth & White & White & Partial & White & One & Pendant & Black & Numerous & Grasses \\ 
   \hline
\end{tabular}
}
\caption{test} 
\label{tab:example}
\end{table}


The good thing about the above approach is you can always refer to the table in your document. For example 'As can be seen in Table~\ref{tab1},...'

\textbf{Figures:} are important part of your analysis, and also good way to give insight about the dataset you are working with. You will use packages like \textit{ggplot2} to produce some visuals. Lets start with a simple example to show how to present your visuals in the report with proper labels and captions. Remember, I need to see the code the produced the figure as well. \\

Suppose, I just want to create a plot that shows the relation between Petal width and length and map the colour and shape to the Species (class label in my dataset). \\

First, I will write my code, but notice that my chunk definition is set as follows\\ \textit{(<<warning=FALSE,message=FALSE,eval=FALSE>>=)}, I set warning and message to FALSE, because I don't want these warnings/ messages to appear in my final output. 


\begin{knitrout}
\definecolor{shadecolor}{rgb}{0.969, 0.969, 0.969}\color{fgcolor}\begin{kframe}
\begin{alltt}
\hlkwd{library}\hlstd{(ggplot2)}

\hlstd{p} \hlkwb{<-} \hlkwd{ggplot}\hlstd{(iris,} \hlkwd{aes}\hlstd{(}\hlkwc{x}\hlstd{=Petal.Length,}
                      \hlkwc{y} \hlstd{= Petal.Width,}\hlkwc{color}\hlstd{=Species) )}
\hlstd{p} \hlkwb{<-} \hlstd{p}\hlopt{+} \hlkwd{geom_point}\hlstd{(}\hlkwd{aes}\hlstd{(}\hlkwc{shape}\hlstd{=Species))}
\hlstd{p} \hlkwb{<-} \hlstd{p} \hlopt{+} \hlkwd{xlab}\hlstd{(}\hlstr{'Petal Length'}\hlstd{)}
\hlstd{p} \hlkwb{<-} \hlstd{p} \hlopt{+} \hlkwd{ylab}\hlstd{(}\hlstr{'Petal Width'}\hlstd{)}
\hlstd{p} \hlkwb{<-} \hlstd{p} \hlopt{+} \hlkwd{theme_bw}\hlstd{()}
\hlstd{p}
\end{alltt}
\end{kframe}
\end{knitrout}

Make sure that your code is running, and once everything is OK, then you need to insert the above code within a Latex code used to insert images (check the .rnw file to see how we achieved this). Again, remember the caption and the label which allows you to refer to this figure from anywhere in your document (Figure~\ref{fig1}). Notice the header of the chunk code in the \textit{.rnw} file. 

\begin{figure}[H]
\begin{center}

\begin{knitrout}
\definecolor{shadecolor}{rgb}{0.969, 0.969, 0.969}\color{fgcolor}
\includegraphics[width=.76\linewidth]{figure/unnamed-chunk-17-1} 

\end{knitrout}
\caption {Petal Length /Width per species in IRIS set}
\label{fig1}
\end {center}
\end {figure}



\section {Modeling and Classifcation}



\subsection {Divide into training and}


\subsection{Build Classifier}


Complete this part as required by the coursework sheet. Again, be clear, visuals always helps in communicating results. Justify your choices and explain your methods. 














% Clear the page and starte a new page for references 

\clearpage
% The title for the reference section is called References 

\section{Appendix}\label{pubs}

\subsection{Mushroom Dataset Helper Function}

I used a helper function to import the dataset, it helps with assigning the correct column and row names to the dataset. It also removes any missing values from the dateset.


\lstset{ 
  language=R,                     % the language of the code
  basicstyle=\tiny\ttfamily, % the size of the fonts that are used for the code
  stepnumber=1,                   % the step between two line-numbers. If it is 1, each line
                                  % will be numbered
  numbersep=5pt,                  % how far the line-numbers are from the code
  backgroundcolor=\color{white},  % choose the background color. You must add \usepackage{color}
  showspaces=false,               % show spaces adding particular underscores
  showstringspaces=false,         % underline spaces within strings
  showtabs=false,                 % show tabs within strings adding particular underscores
  frame=single,                   % adds a frame around the code
  rulecolor=\color{black},        % if not set, the frame-color may be changed on line-breaks within not-black text (e.g. commens (green here))
  tabsize=2,                      % sets default tabsize to 2 spaces
  captionpos=b,                   % sets the caption-position to bottom
  breaklines=true,                % sets automatic line breaking
  breakatwhitespace=false,        % sets if automatic breaks should only happen at whitespace
 keywordstyle=\color{RoyalBlue},      % keyword style
  commentstyle=\color{YellowGreen},   % comment style
  stringstyle=\color{ForestGreen}      % string literal style
} 



% \lstset{ %
% basicstyle=\footnotesize,       % the size of the fonts that are used for the code
% numbers=left,           % where to put the line-numbers
% numberstyle=\footnotesize,  % the size of the fonts that are used for the line-numbers
% stepnumber=1,           % the step between two line-numbers. If it is 1 each line will be numbered
% numbersep=5pt,          % how far the line-numbers are from the code
% backgroundcolor=\color{white},  % choose the background color. You must add \usepackage{color}
% showspaces=false,           % show spaces adding particular underscores
% showstringspaces=false,     % underline spaces within strings
% showtabs=false,         % show tabs within strings adding particular underscores
% frame=single,           % adds a frame around the code
% tabsize=2,              % sets default tabsize to 2 spaces
% captionpos=b,           % sets the caption-position to bottom
% breaklines=true,            % sets automatic line breaking
% breakatwhitespace=false,        % sets if automatic breaks should only happen at whitespace
% escapeinside={\%*}{*)},      % if you want to add a comment within your code
% breaklines=true,
%   postbreak=\mbox{\textcolor{red}{$\hookrightarrow$}\space}
% }

    \begin{lstlisting}
fetchAndCleanData = function(){
  
  # All of this code is from
  # https://rstudio-pubs-static.s3.amazonaws.com/125760_358e4a6802c94fa29e2a9ab49f45df94.html
  
  mushrooms = read.table("data/agaricus-lepiota.data", header = FALSE, sep = ",")
  
  
  
  #create a data frame with only the required columns
  shrooms = mushrooms
  
  #column names are added
  colnames(shrooms) = c("Edible",
                        "CapShape",
                        "CapSurface",
                        "CapColor",
                        "Bruises",                        
                        "Odor",
                        "GillAttachment",
                        "GillSpacing",
                        "GillSize",
                        "GillColor",
                        "StalkShape",
                        "StalkRoot",
                        "StalkSurfaceAboveRing",
                        "StalkSurfaceBelowRing",
                        "StalkColorAboveRing",
                        "StalkColorBelowRing",
                        "VeilType",
                        "VeilColor",
                        "RingNumber",
                        "RingType",
                        "SporePrintColor",
                        "Population",
                        "Habitat")
  
  
  #Edible
  shrooms$Edible = as.character(shrooms$Edible)
  shrooms$Edible[shrooms$Edible == "e"] = "Edible"
  shrooms$Edible[shrooms$Edible == 'p'] = "Poisonous"
  shrooms$Edible = factor(shrooms$Edible)
  
  
  # Edible
  #levels(shrooms$Edible) = c(levels(shrooms$Edible), c("Poisonous","Edible"))
  #shrooms$Edible[shrooms$Edible == "p"] = "Poisonous"
  #shrooms$Edible[shrooms$Edible == "e"] = "Edible"
  
  #CapShape
  levels(shrooms$`CapShape`) = c(levels(shrooms$`CapShape`), c("Bell","Conical","Convex","Flat","Knobbed","Sunken"))
  shrooms$`CapShape`[shrooms$`CapShape` == "b"] = "Bell"
  shrooms$`CapShape`[shrooms$`CapShape` == "c"] = "Conical"
  shrooms$`CapShape`[shrooms$`CapShape` == "x"] = "Convex"
  shrooms$`CapShape`[shrooms$`CapShape` == "f"] = "Flat"
  shrooms$`CapShape`[shrooms$`CapShape` == "k"] = "Knobbed"
  shrooms$`CapShape`[shrooms$`CapShape` == "s"] = "Sunken"
  
  #CapSurface
  levels(shrooms$`CapSurface`) = c(levels(shrooms$`CapSurface`), c("Fibrous", "Grooves", "Scaly", "Smooth"))
  shrooms$`CapSurface`[shrooms$`CapSurface` == "f"] = "Fibrous"
  shrooms$`CapSurface`[shrooms$`CapSurface` == "g"] = "Grooves"
  shrooms$`CapSurface`[shrooms$`CapSurface` == "y"] = "Scaly"
  shrooms$`CapSurface`[shrooms$`CapSurface` == "s"] = "Smooth"
  
  #CapColor
  levels(shrooms$`CapColor`) = c(levels(shrooms$`CapColor`), c("Brown", "Buff", "Cinnamon", "Gray", "Green", "Pink", "Purple", "Red", "White", "Yellow"))
  shrooms$`CapColor`[shrooms$`CapColor` == "n"] = "Brown"
  shrooms$`CapColor`[shrooms$`CapColor` == "b"] = "Buff"
  shrooms$`CapColor`[shrooms$`CapColor` == "c"] = "Cinnamon"
  shrooms$`CapColor`[shrooms$`CapColor` == "g"] = "Gray"
  shrooms$`CapColor`[shrooms$`CapColor` == "r"] = "Green"
  shrooms$`CapColor`[shrooms$`CapColor` == "p"] = "Pink"
  shrooms$`CapColor`[shrooms$`CapColor` == "u"] = "Purple"
  shrooms$`CapColor`[shrooms$`CapColor` == "e"] = "Red"
  shrooms$`CapColor`[shrooms$`CapColor` == "w"] = "White"
  shrooms$`CapColor`[shrooms$`CapColor` == "y"] = "Yellow"
  
  # Bruises
  levels(shrooms$Bruises) = c(levels(shrooms$Bruises), c("True","False"))
  shrooms$Bruises[shrooms$Bruises == "t"] = "True"
  shrooms$Bruises[shrooms$Bruises == "f"] = "False"
  
  #Odor
  levels(shrooms$Odor) = c(levels(shrooms$Odor), c("Almond", "Anise", "Creosote", "Fishy", "Foul", "Musty", "None", "Pungent", "Spicy"))
  shrooms$Odor[shrooms$Odor == "a"] = "Almond"
  shrooms$Odor[shrooms$Odor == "l"] = "Anise"
  shrooms$Odor[shrooms$Odor == "c"] = "Creosote"
  shrooms$Odor[shrooms$Odor == "y"] = "Fishy"
  shrooms$Odor[shrooms$Odor == "f"] = "Foul"
  shrooms$Odor[shrooms$Odor == "m"] = "Musty"
  shrooms$Odor[shrooms$Odor == "n"] = "None"
  shrooms$Odor[shrooms$Odor == "p"] = "Pungent"
  shrooms$Odor[shrooms$Odor == "s"] = "Spicy"
  
  # GillAttachment
  levels(shrooms$GillAttachment) = c(levels(shrooms$GillAttachment), c("Attached","Descending","Free","Notched"))
  shrooms$GillAttachment[shrooms$GillAttachment == "a"] = "Attached"
  shrooms$GillAttachment[shrooms$GillAttachment == "d"] = "Descending"
  shrooms$GillAttachment[shrooms$GillAttachment == "f"] = "Free"
  shrooms$GillAttachment[shrooms$GillAttachment == "n"] = "Notched"
  
  # GillSpacing
  levels(shrooms$GillSpacing) = c(levels(shrooms$GillSpacing), c("Close","Crowded","Distant"))
  shrooms$GillSpacing[shrooms$GillSpacing == "c"] = "Close"
  shrooms$GillSpacing[shrooms$GillSpacing == "w"] = "Crowded"
  shrooms$GillSpacing[shrooms$GillSpacing == "d"] = "Distant"
  
  # GillSize
  levels(shrooms$GillSize) = c(levels(shrooms$GillSize), c("Broad","Narrow"))
  shrooms$GillSize[shrooms$GillSize == "b"] = "Broad"
  shrooms$GillSize[shrooms$GillSize == "n"] = "Narrow"
  
  # GillColor
  levels(shrooms$GillColor) = c(levels(shrooms$GillColor), c("Black","Brown","Buff","Chocolate","Gray","Green","Orange","Pink","Purple","Red","White","Yellow"))
  shrooms$GillColor[shrooms$GillColor == "k"] = "Black"
  shrooms$GillColor[shrooms$GillColor == "n"] = "Brown"
  shrooms$GillColor[shrooms$GillColor == "b"] = "Buff"
  shrooms$GillColor[shrooms$GillColor == "h"] = "Chocolate"
  shrooms$GillColor[shrooms$GillColor == "g"] = "Gray"
  shrooms$GillColor[shrooms$GillColor == "r"] = "Green"
  shrooms$GillColor[shrooms$GillColor == "o"] = "Orange"
  shrooms$GillColor[shrooms$GillColor == "p"] = "Pink"
  shrooms$GillColor[shrooms$GillColor == "u"] = "Purple"
  shrooms$GillColor[shrooms$GillColor == "e"] = "Red"
  shrooms$GillColor[shrooms$GillColor == "w"] = "White"
  shrooms$GillColor[shrooms$GillColor == "y"] = "Yellow"
  
  # StalkShape
  levels(shrooms$StalkShape) = c(levels(shrooms$StalkShape), c("Enlarging","Tapering"))
  shrooms$StalkShape[shrooms$StalkShape == "e"] = "Enlarging"
  shrooms$StalkShape[shrooms$StalkShape == "t"] = "Tapering"
  
  # StalkRoot
  levels(shrooms$StalkRoot) = c(levels(shrooms$StalkRoot), c("Bulbous","Club","Cup","Equal","Rhizomorphs","Rooted","Missing"))
  shrooms$StalkRoot[shrooms$StalkRoot == "b"] = "Bulbous"
  shrooms$StalkRoot[shrooms$StalkRoot == "c"] = "Club"
  shrooms$StalkRoot[shrooms$StalkRoot == "u"] = "Cup"
  shrooms$StalkRoot[shrooms$StalkRoot == "e"] = "Equal"
  shrooms$StalkRoot[shrooms$StalkRoot == "z"] = "Rhizomorphs"
  shrooms$StalkRoot[shrooms$StalkRoot == "r"] = "Rooted"
  shrooms$StalkRoot[shrooms$StalkRoot == "?"] = "Missing"
  
  # StalkSurfaceAboveRing
  levels(shrooms$StalkSurfaceAboveRing) = c(levels(shrooms$StalkSurfaceAboveRing), c("Fibrous","Scaly","Silky","Smooth"))
  shrooms$StalkSurfaceAboveRing[shrooms$StalkSurfaceAboveRing == "f"] = "Fibrous"
  shrooms$StalkSurfaceAboveRing[shrooms$StalkSurfaceAboveRing == "y"] = "Scaly"
  shrooms$StalkSurfaceAboveRing[shrooms$StalkSurfaceAboveRing == "k"] = "Silky"
  shrooms$StalkSurfaceAboveRing[shrooms$StalkSurfaceAboveRing == "s"] = "Smooth"
  
  # StalkSurfaceBelowRing
  levels(shrooms$StalkSurfaceBelowRing) = c(levels(shrooms$StalkSurfaceBelowRing), c("Fibrous","Scaly","Silky","Smooth"))
  shrooms$StalkSurfaceBelowRing[shrooms$StalkSurfaceBelowRing == "f"] = "Fibrous"
  shrooms$StalkSurfaceBelowRing[shrooms$StalkSurfaceBelowRing == "y"] = "Scaly"
  shrooms$StalkSurfaceBelowRing[shrooms$StalkSurfaceBelowRing == "k"] = "Silky"
  shrooms$StalkSurfaceBelowRing[shrooms$StalkSurfaceBelowRing == "s"] = "Smooth"
  
  # StalkColorAboveRing
  levels(shrooms$StalkColorAboveRing) = c(levels(shrooms$StalkColorAboveRing), c("Brown","Buff","Cinnamon","Gray","Orange","Pink","Red","White","Yellow"))
  shrooms$StalkColorAboveRing[shrooms$StalkColorAboveRing == "n"] = "Brown"
  shrooms$StalkColorAboveRing[shrooms$StalkColorAboveRing == "b"] = "Buff"
  shrooms$StalkColorAboveRing[shrooms$StalkColorAboveRing == "c"] = "Cinnamon"
  shrooms$StalkColorAboveRing[shrooms$StalkColorAboveRing == "g"] = "Gray"
  shrooms$StalkColorAboveRing[shrooms$StalkColorAboveRing == "o"] = "Orange"
  shrooms$StalkColorAboveRing[shrooms$StalkColorAboveRing == "p"] = "Pink"
  shrooms$StalkColorAboveRing[shrooms$StalkColorAboveRing == "e"] = "Red"
  shrooms$StalkColorAboveRing[shrooms$StalkColorAboveRing == "w"] = "White"
  shrooms$StalkColorAboveRing[shrooms$StalkColorAboveRing == "y"] = "Yellow"
  
  
  # StalkColorBelowRing
  levels(shrooms$StalkColorBelowRing) = c(levels(shrooms$StalkColorBelowRing), c("Brown","Buff","Cinnamon","Gray","Orange","Pink","Red","White","Yellow"))
  shrooms$StalkColorBelowRing[shrooms$StalkColorBelowRing == "n"] = "Brown"
  shrooms$StalkColorBelowRing[shrooms$StalkColorBelowRing == "b"] = "Buff"
  shrooms$StalkColorBelowRing[shrooms$StalkColorBelowRing == "c"] = "Cinnamon"
  shrooms$StalkColorBelowRing[shrooms$StalkColorBelowRing == "g"] = "Gray"
  shrooms$StalkColorBelowRing[shrooms$StalkColorBelowRing == "o"] = "Orange"
  shrooms$StalkColorBelowRing[shrooms$StalkColorBelowRing == "p"] = "Pink"
  shrooms$StalkColorBelowRing[shrooms$StalkColorBelowRing == "e"] = "Red"
  shrooms$StalkColorBelowRing[shrooms$StalkColorBelowRing == "w"] = "White"
  shrooms$StalkColorBelowRing[shrooms$StalkColorBelowRing == "y"] = "Yellow"
  
  # VeilType
  levels(shrooms$VeilType) = c(levels(shrooms$VeilType), c("Partial","Universal"))
  shrooms$VeilType[shrooms$VeilType == "p"] = "Partial"
  shrooms$VeilType[shrooms$VeilType == "u"] = "Universal"
  
  # VeilColor
  levels(shrooms$VeilColor) = c(levels(shrooms$VeilColor), c("Brown","Orange","White","Yellow"))
  shrooms$VeilColor[shrooms$VeilColor == "n"] = "Brown"
  shrooms$VeilColor[shrooms$VeilColor == "o"] = "Orange"
  shrooms$VeilColor[shrooms$VeilColor == "w"] = "White"
  shrooms$VeilColor[shrooms$VeilColor == "y"] = "Yellow"
  
  # RingNumber
  levels(shrooms$RingNumber) = c(levels(shrooms$RingNumber), c("None","One","Two"))
  shrooms$RingNumber[shrooms$RingNumber == "n"] = "None"
  shrooms$RingNumber[shrooms$RingNumber == "o"] = "One"
  shrooms$RingNumber[shrooms$RingNumber == "t"] = "Two"
  
  # RingType
  levels(shrooms$RingType) = c(levels(shrooms$RingType), c("Cobwebby","Evanescent","Flaring","Large","None","Pendant","Sheathing","Zone"))
  shrooms$RingType[shrooms$RingType == "c"] = "Cobwebby"
  shrooms$RingType[shrooms$RingType == "e"] = "Evanescent"
  shrooms$RingType[shrooms$RingType == "f"] = "Flaring"
  shrooms$RingType[shrooms$RingType == "l"] = "Large"
  shrooms$RingType[shrooms$RingType == "n"] = "None"
  shrooms$RingType[shrooms$RingType == "p"] = "Pendant"
  shrooms$RingType[shrooms$RingType == "s"] = "Sheathing"
  shrooms$RingType[shrooms$RingType == "z"] = "Zone"
  
  # SporePrintColor
  levels(shrooms$SporePrintColor) = c(levels(shrooms$SporePrintColor), c("Black","Brown","Buff","Chocolate","Green","Orange","Purple","White","Yellow"))
  shrooms$SporePrintColor[shrooms$SporePrintColor == "k"] = "Black"
  shrooms$SporePrintColor[shrooms$SporePrintColor == "n"] = "Brown"
  shrooms$SporePrintColor[shrooms$SporePrintColor == "b"] = "Buff"
  shrooms$SporePrintColor[shrooms$SporePrintColor == "h"] = "Chocolate"
  shrooms$SporePrintColor[shrooms$SporePrintColor == "r"] = "Green"
  shrooms$SporePrintColor[shrooms$SporePrintColor == "o"] = "Orange"
  shrooms$SporePrintColor[shrooms$SporePrintColor == "u"] = "Purple"
  shrooms$SporePrintColor[shrooms$SporePrintColor == "w"] = "White"
  shrooms$SporePrintColor[shrooms$SporePrintColor == "y"] = "Yellow"
  
  # Population
  levels(shrooms$Population) = c(levels(shrooms$Population), c("Abundnant","Clustered","Numerous","Scattered","Several","Solitary"))
  shrooms$Population[shrooms$Population == "a"] = "Abundnant"
  shrooms$Population[shrooms$Population == "c"] = "Clustered"
  shrooms$Population[shrooms$Population == "n"] = "Numerous"
  shrooms$Population[shrooms$Population == "s"] = "Scattered"
  shrooms$Population[shrooms$Population == "v"] = "Several"
  shrooms$Population[shrooms$Population == "y"] = "Solitary"
  
  # Habitat
  levels(shrooms$Habitat) = c(levels(shrooms$Habitat), c("Grasses","Leaves","Meadows","Paths","Urban","Waste","Woods"))
  shrooms$Habitat[shrooms$Habitat == "g"] = "Grasses"
  shrooms$Habitat[shrooms$Habitat == "l"] = "Leaves"
  shrooms$Habitat[shrooms$Habitat == "m"] = "Meadows"
  shrooms$Habitat[shrooms$Habitat == "p"] = "Paths"
  shrooms$Habitat[shrooms$Habitat == "u"] = "Urban"
  shrooms$Habitat[shrooms$Habitat == "w"] = "Waste"
  shrooms$Habitat[shrooms$Habitat == "d"] = "Woods"
  
  return(shrooms)
}

\end{lstlisting}




% Clear the page and starte a new page for references 

\clearpage
% The title for the reference section is called References 

\section{References}\label{pubs}

\printbibliography[heading =none]


\clearpage


\end{document}
